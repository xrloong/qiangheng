\chapter{原碼}

目錄結構如下:

\begin{minipage}{4cm}\dirtree{%
.1 qiangheng.py.
.1 canvas/.
.1 description/.
.1 gear/.
.1 hanzi.
.1 im/.
.2 base/.
.2 Array/.
.2 Boshiamy/.
.2 CangJie/.
.2 DaYi/.
.2 DynamicComposition/.
.2 FourCorner/.
.2 GuiXie/.
.2 Sample/.
.2 StrokeOrder/.
.2 ZhengMa/.
.1 parser/.
.1 state/.
.1 util/.
}\end{minipage}


其中,qiangheng.py 為主程式,是程式的進入點。

要描述一個輸入法,需要定義:
\begin{enumerate}
\item[一、]IMInfo
用於描述一個輸入法的資訊。

\item[二、]CodeInfo
一個輸入法為一個字符編碼所應包含的資訊。

\item[三、]CodeInfoEncoder
用於為不同結構的組字進行編碼。

\item[四、]RadixParser
各輸入法的剖析器。

\item[五、]StructureRearranger
用來為自行結構重新排列。

\end{enumerate}

