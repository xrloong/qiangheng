\documentclass{article}
\usepackage{CJKnumb}

\usepackage{fontspec}
\setmainfont[Mapping=tex-text]{AR PL UMing TW}
%\setmainfont[Mapping=tex-text]{文泉驛正黑}
\XeTeXlinebreaklocale ``zh''
\XeTeXlinebreakskip = 0pt plus 1pt

\usepackage{amsmath}
\usepackage{multirow}

\def\tac{\textasciicircum}
\title{瑲珩}
\author{王湘叡}

\newcommand\qhchar[1]{\mbox{#1}}

\newcommand\qhstroke[1]{\qhchar{#1}_{\mbox{筆}}}

\newcommand\qhar[1]{\qhchar{#1}_{\mbox{行}}}
\newcommand\qharrule{\mbox{取}_{\mbox{前三後一}}}
\newcommand\qhdy[1]{\qhchar{#1}_{\mbox{易}}}
\newcommand\qhdyrule{\mbox{取}_{\mbox{前三後一}}}
\newcommand\qhbs[1]{\qhchar{#1}_{\mbox{嘸}}}
\newcommand\qhbscomp[1]{\qhchar{#1}_{\mbox{嘸補}}}
\newcommand\qhbstemp[1]{\qhchar{#1}_{\mbox{嘸暫}}}
\newcommand\qhzm[1]{\qhchar{#1}_{\mbox{鄭}}}
\newcommand\qhzmlist[1]{\qhchar{#1}_{\mbox{鄭串}}}
\newcommand\qhcj[1]{\qhchar{#1}_{\mbox{倉}}}
\newcommand\qhcjlist[1]{\qhchar{#1}_{\mbox{倉串}}}
\newcommand\qhcjdir[1]{\qhchar{#1}_{\mbox{倉向}}}
\newcommand\qhcjbody[1]{\qhchar{#1}_{\mbox{倉身}}}
\newcommand\qhcjmerge[2]{\oplus_{\qhchar{#1}}(#2)}

\begin{document}
\maketitle{}

漢字系統,是一個特別的書寫系統。%
它利用數千以至上萬種不同的字形來表達各種意義。%
然而,這些字形卻可以由數百至一千多種字形所構成。
具體的構造方法,則闡述於漢字構造的「六書」中的「會意」及「形聲」。

因為可以藉由組合的方式產生新字,%
於是就產生了兩個特性:「開放」、「數量龐大」。%
「開放」意味著可以依需求造字,%
千餘年來不斷有新的字被造出來,%
這又導致了「數量龐大」的這個特性。%
目前累積已有約十萬字(包含日本、韓國及越南所造漢字)。%
而這兩個特性則都對漢字的資訊化產生了了一定程度的影響。

通常一種語言要資訊化,要經過以下的步驟:
\begin{enumerate}
\item[一、]搜集該語言所用到的字形。\\
搜集到的字形所成的集合稱做字符集或字集。%
但因漢字的開放特性,不可能將所有字都搜集。

\item[二、]賦予每個字形一個獨立的編號(專業的術語稱作編碼)。\\
因為電腦並擅長處理的是數值而非二維的圖形,所以需要對每個字形編號。%
於是,原本對人而言的文字的處理,如儲存、搜尋、排序或比對,%
對電腦而言就是儲存、搜尋、排序或比對數值。%
但因為不可能搜集到所有的字,意味著漏掉的字將無法資訊化。

\item[三、]產生字形檔,要為每個字形準備相對應的圖。\\
電腦可以很輕易地處理數值,但人不行。%
因此,當電腦處理完數值後,要將結果呈現給人時,需要轉換成人類所能理解的形式。%
由於「數量龐大」的原因,要為漢字產生一種字型往往需要耗費大量的人力。%
而如果字符集改變,如新造了或搜集到了一個字,則每種字型再造形。%
為了能將大量的工作轉為自動化,就有了「動態組字」的研究出現。%
「動態組字」的目標在於,利用漢字組字的特性,能動態的將字形產生出來。%
即我們只要準備一些字根的字形,而電腦自動幫我們產生所有漢字的字形。

\item[四、]定義輸入法,為每個字形生成輸入碼。\\
大部分語言所定義的字形數都不像漢字系統採取如此龐大的數量,%
因此在輸入時,通常是使用特殊的鍵盤來容納所有字形,且按壓一個鍵將產生一個字形。
然而漢字系統無法採用此種方式,因此許多字形輸入法被開發出來。%
亦即,利用一組「按鍵序列」來產生一個字。%
然而,因為「數量龐大」的特性,要為一個字形產生一組輸入法,往往也要耗費大量的人力。%
又如新造了或搜集到了一個字,則每種字形都要再次為此字形拆碼。%
為了能將大量的工作轉為自動化,就是本計劃的目標,可以視為「動態拆碼」。

\end{enumerate}

\section{原理}
在開始之前,讓我們可以利用漢字的``筆劃數''的這個屬性來做說明。%
假設我們需要知道``曉''這個字的筆劃數,%
最直覺得方法就是:從第一筆(日的第一筆)到最後一筆(兀的最後一筆),%
一筆一筆地去描繪,在描繪的同時邊計數,最後得到共十六劃,這樣就完成了任務。%

這個方法雖然直覺而且簡單,%
但若需要算出的不是一``個''字的筆劃數,而是一``羣''字(如所有漢字)的筆劃,這個方法就顯得笨拙。
首先,我們需要花費大量的人力去做這件事,而且過程中極易出錯。
其次,每個人可能會使用不同的標準(如對同一個字選擇不同的字形或筆順)。
此外,如果在過程中對標準做出修訂,則又要重新花費大量的人力做一樣的事。

之所以會如此的原因在於這方法沒有充分使用到漢字系統的特性。
在漢字系統中,大部分的漢字都是由其它漢字或字根所組合出來的。
於是,新的漢字具有字根的特性。

如果考慮了這個特性,我們可以採用另一個方法:\\
因為\\
\[
  \qhchar{``曉''}=\qhchar{``日''}+\qhchar{``堯''}
\]
若已經知道``日''為四劃而``堯''為十二劃,則只要用加法就可得到``曉''為十六劃了。%

然而一個問題是:要怎麼才可以知道``日''、``堯''的筆劃數呢?%
我們一樣可以採用類似的方法。
因為\\
\[
  \qhchar{``堯''}=\qhchar{``垚''}+\qhchar{``兀''}
\]
也就是如果事先已經知道``垚''為九劃和``兀''為三劃,則用加法就可得到十二劃。
同樣地,考慮到\\
\[
  \qhchar{``垚''}=\qhchar{``土''}\times 3
\]
也就是,如果事先已經知道``土''為三劃,則只要用乘法就可得到``垚''為九劃。

於是,最後的問題變成:只要知道``土''、``兀''和``日''的筆劃數,我們就可以算出``曉''的筆劃數。
而要知道``土''、``兀''和``日''的筆劃數,則只能用一筆劃一筆劃去地去計算。

從另一個角度來想:
只要我們用一筆劃一筆劃地去計算``土''、``兀''和``日''的筆劃數,
而不用一筆劃一筆劃地去描繪``曉'',
我們就可以計算出``曉''的筆劃數。

實際上,對求出``曉''的筆劃這個問題而言,
這個方法不會比一筆劃一筆劃地去計算來得快,
(因為要一筆劃一筆劃地去計算``土''、``兀''和``日''的筆劃數,且還要做一些加法或乘法運算)。

如果現在的情況是``要計算一堆字的筆劃數'',則這個方法可以大幅減少工作量及時間。

考量到很多部件在中文字是時常出現的,如``土''、``日''等,
以上面例子來說,只要知道``土''、``兀''和``日''的筆劃數,
我們不但在計算過程中得到了``垚'',``堯'',``曉''的筆劃數。
只要再加上一些計算,
我們同樣可以算出``昌'',``昍'',``晶'',``晿'',``圭'',``圼'',``晆'',``曉''等字的筆劃數。


此外,這個方法也適合自動化。
只要知道每個漢字的組成方式,並且有一些基礎字根的資料。
我們就可以自動算出全部的值。
自動化的好處還有標準統一,如果有一些標準改了,也可快速重新計算。

\begin{subequations}
  \begin{align}
  \qhstroke{字} &= \text{筆劃數}\\
  \qhstroke{丙} &= \qhstroke{甲} + \qhstroke{乙}\
  \end{align}
\end{subequations}

如果我們的目標不是計算``筆劃數'',而是創造字形,這個方法一樣可以適用,這就是``動態組字''的範疇。
如果我們的目標不是計算``筆劃數'',而是輸入碼,這個方法一樣可以適用,這就是本計劃的範疇。

如果我們將類似的原理套用在輸入法上:
只要我們事先知道一些基本或不易分割的部件的外碼(字碼),
並在結合部件時,採用一定的方式去組合外碼,就可以算出那個字的外碼。
我們就可以省下大量的功夫,甚至是用電腦來計算。
%``曉''=``日''+``堯''\\
%``堯''=``垚''+``兀''\\
%``垚''=``土''+``土''+``土''\\
%%``土'',``日'',``垚'',``堯'',``曉'',''兀''
%%``土'',``垚'',``堯'',``曉'',``兀''

本計劃目前選擇了五種字形輸入法:倉頡、行列、嘸蝦米、大易、鄭碼。

\section{用辭說明}
\begin{itemize}
\item 首碼\\
字根的第一個碼。以``靣''來說,在倉頡中拆作``一田口'',首碼即為一\\
\item 次碼\\
字根的第二個碼。以``靣''來說,在倉頡中拆作``一田口'',首碼即為田\\
\item 三碼\\
字根的第三個碼。以``靣''來說,在倉頡中拆作``一田口'',首碼即為口田\\
\item 末碼\\
字根的最後一個碼。以``靣''來說,在倉頡中拆作``一田口'',首碼即為口\\
\item 尾碼\\
在倉頡中,有時取碼時,並不是取字根的最後一碼,而是最後中的特徵碼。\\
為了與末碼區分,稱之為尾碼。以``靣''來說,在倉頡中拆作``一田口'',首碼即為田
 
\item 簡碼\\
\item 快碼\\
將較常出現的字以較短的編碼來指定者稱之。
將較常出現的字根以較短的編碼來指定者稱之。
如,

\item 標準編碼\\
根據輸入法規則而得到的編碼。

\item 容錯編碼\\
輸入法為了讓使用者有更好的體驗,
為了預防使用者選用不同的字集,為拆碼見解不同於標準,
所以提供多種編碼。

\item 多碼\\
一個字在同一種輸入法下可以有兩種以上的編碼。不同於容錯編碼的概念。\\
      容錯編碼基本上是不標準的碼,如字形不同。但此指的都是標準編碼。
      如注音,一個字可以有很多種唸法,於是,就有多種編碼。
      只要符合規則的,即為標準編碼。
      多碼與容錯編碼間有模糊地帶。

%\item 重碼\\
%不同的字,卻有相同的編碼,稱為重碼。重碼的比率稱為重碼率。\\
%
\end{itemize}


\section{行列輸入法、大易輸入法}
\subsection{輸入法說明}
行列與大易輸入法,看起來像是兩個截然不同的輸入法(事實上,學習方式也非常地不同)。%
然而,兩者除了所選用的字根及排列方式不同外,
就規則而言,是頗為相像的。
兩者的規則皆為前三後一,
即若一個字拆成字根後,字根個數小於四個則全取。
大於等於四個,則取首碼、次碼、三碼和末碼。

同樣也是以``曉''為例。\\
\begin{tabular}{llll}
字  & 行列碼 & 大易碼\\
日  & P(0\tac) & D\\
土  & R(4\tac) & F\\
垚  & RRR(4\tac4\tac4\tac) & FFF\\
兀  & AS(1-2-) & EQ\\
堯  & RRRS(4\tac4\tac4\tac2-) & FFFQ\\
曉  & PRRS(0\tac4\tac4\tac2-) & DFFQ\\
\end{tabular}

\subsection{遞迴式}
分別使用$\qhar{字}$和$\qhdy{字}$來表示一個字的行列碼與大易碼。\\
如 $\qhar{曉}$ =``PRRS''、$\qhdy{曉}$=``DFFQ''。
\begin{subequations}
  \begin{align}
  \qhar{字} &= 行列碼\\
  \qhdy{字} &= 大易碼\\
  甲 \oplus 乙 &= 取(甲+乙)的前三後一碼
  \end{align}
\end{subequations}

若$(\qhchar{丙}=\qhchar{甲}+\qhchar{乙})$
其遞迴算式為:
\begin{subequations}
  \begin{align}
  \qhar{丙}&=\qharrule{}(\qhar{甲} \oplus \qhar{乙})\\
  \qhdy{丙}&=\qhdyrule{}(\qhdy{甲} \oplus \qhdy{乙})
  \end{align}
\end{subequations}

\subsection{資料格式說明}
\subsection{注意事項}
對於行列輸入法而言,``囚''即為特別,它有兩個拆法。
在``溫''中及在``囚''中的拆法不一。照行列的說法是以面積來決定。

\section{嘸蝦米輸入法}
\subsection{輸入法說明}
嘸蝦米的規則跟行列與大易很相像。同樣為前三後一。
特別的是,嘸蝦米多了一個補碼規定--若取碼不足兩碼,則要根據最後一筆劃添加補碼。
為此,為嘸蝦米添加一個屬性:
用``嘸''、``.嘸補'' 表示一個字的嘸蝦米碼和補碼,\\
\subsection{遞迴式}
使用``$\qhbscomp{字}$''來表示一個字的嘸蝦米補碼。
使用``$\qhbs{字}$''來表示一個字的嘸蝦米碼。
$\qhbs{垚}=YYY$\\
$\qhbscomp{兀}=L$\\
則
\begin{subequations}
  \begin{align}
  \qhbs{字} &= 嘸蝦米碼\\
  \qhbscomp{字} &= 嘸蝦米補碼\\
  \qhbstemp{字} &= 嘸蝦米暫時碼,即沒有補碼\\
  \qhbs{字} &=
      \left\{\begin{array}{ll}
        \qhbstemp{字}
           & \text{若$\qhbstemp{字} \geq $三碼}\\
        \qhbstemp{字}+\qhbscomp{字}
           & \text{若$\qhbstemp{丙} \leq $二碼}
      \end{array}\right.\\
  甲 \oplus 乙 &= 取(甲+乙)的前三後一碼\\
  \end{align}
\end{subequations}

若$(\qhchar{丙}=\qhchar{甲}+\qhchar{乙})$
其遞迴算式為:
\begin{subequations}
  \begin{align}
  \qhbscomp{丙}&=\qhbscomp{乙}\\
  \qhbstemp{丙}&=\qhbstemp{甲} \oplus \qhbscomp{乙}\\
  \end{align}
\end{subequations}

\subsection{資料格式說明}
\subsection{注意事項}

\section{鄭碼輸入法}
\subsection{輸入法說明}
鄭碼的規則有點複雜。首先,鄭碼會定義一些字根,並為每個字根編碼。
到目前為止,還跟其它輸入法類似。不同的是,鄭碼會使用二碼來為字根編碼。

鄭碼的字根大多為二碼字,對一些常用的部件則為一碼。
在少數的情況下,有三碼。
鄭碼會依不同的情況,可以省略一些部件或位碼。
須要計算部件數,並依其個數,來決定規則。
不過,鄭碼會用到的字根最多只有頭兩個和尾兩個,總共最多四個。

\subsection{遞迴式}
使用``$\qhzmlist{字}$''來表示一個字的字根串列。
如$\qhzmlist{曉}=[\qhchar{``日''}, \qhchar{``土''}, \qhchar{``土''}, \qhchar{兀}]$\\

\begin{subequations}
  \begin{align}
    \qhzmlist{字} &= 表示一個字的鄭碼字根串列。\\
    甲 \oplus 乙 &= 取(甲+乙)的鄭碼字根串列的前二後二。
  \end{align}
\end{subequations}

若$(\qhchar{丙}=\qhchar{甲}+\qhchar{乙})$
其遞迴算式為:
\begin{subequations}
  \begin{align}
  \qhzmlist{丙} &= \qhzmlist{甲} \oplus \qhzmlist{乙} \\
  \end{align}
\end{subequations}

只要為每個字求出其所用到的字根串列,再套用其規則,即可算出鄭碼。
此外,在鄭碼中,區碼的重要性大於位碼,也就是說,在無法全取字根的碼時,會先保留區碼。
如``由''的碼為``KIA'',在遇到``由''這個字根,且只能取一碼時,就取''K''。
且只能取二碼時,就取''KI''。
且只能取三碼時,就取''KIA''。

\begin{tabular}{lcll}
           & 首根碼數 & 規則 & 例字\\
  \multirow{3}{*}{二基根字} & 1 & 首根一碼+末根三碼,若末根只有一碼,則補``VV''\\
  & 2 & 首根二碼+末根二碼\\
  & 3 & 首根三碼+末根一碼\\
  \multirow{3}{*}{三基根字} & 1 & 首根一碼+次末根一碼+末根二碼\\
  & 2 & 首根二碼+次末根一碼+末根一碼\\
  & 3 & 首根三碼+末根一碼\\
  \multirow{4}{*}{四基根字以上} & 1 & 首根一碼+次根一碼+次末根一碼+末根二碼\\
  & 2 & 首根二碼+次末根一碼+末根一碼\\
  & 3 & 首根三碼+末根一碼\\
\end{tabular}

\section{倉頡輸入法}
\subsection{輸入法說明}
倉頡輸入法算是最為複雜的一個輸入法了。
主要是倉頡輸入法不只考慮字根,還會考慮字的結構。

倉頡的規則分為整體字和組合字。
整體字若不足四碼則全取,否則取首、次、三、尾碼。
若為組合字,字首取首、尾兩碼,字身取首、次、尾三碼。

若字身為組合字,當次字首為一碼時,取次字首和次字身取首、尾兩碼。
否則取次字身取首、尾兩碼和次字身取尾碼。

倉頡的分割,是以視覺上的分割,而非邏輯上的分割,如``順''。
對於熟悉中文字的人,會很直覺地分成``川''和``頁''。
但倉頡則是分成``丨''及剩下的部分(即`丨丨頁')。

此外,還要考慮字身的方向性。可分水平、垂直,其它。
如``卲''不分為``刀''和``叩'',而是分為``召''和``阝''。
因為``召''的分向為垂直方向,但``卲''的方向為水平。

\subsection{遞迴式}
\begin{subequations}
  \begin{align}
    \qhcjlist{字} &= 表示一個字的倉頡碼字根串列。\\
    \qhcjdir{字} &= 表示一個字的字根組成方向。\\
    \qhcjbody{字} &= 表示一個字當另一個字的字身時的倉頡碼。\\
  \end{align}
\end{subequations}

如$\qhcjlist{``曉''}=[\qhchar{``日''}, \qhchar{``土''}, \qhchar{``土''}, \qhchar{``山''}]$\\
如$\qhcjdir{``曉''}=`-'$\\
如$\qhcjbody{``曉''}=[\qhchar{``日''}, \qhchar{``土''}, \qhchar{``山''}]$\\
\begin{subequations}
  \begin{align}
    \qhcjmerge{字}{甲} &=
      \left\{\begin{array}{ll}
        \qhcjlist{甲}
           & \text{若$\qhcjdir{甲} = \qhcjdir{字}$}\\
        $[$ \qhcjbody{甲} $]$
           & \text{若$\qhcjbody{甲} \neq \qhcjbody{字}$}
      \end{array}\right.\\
  \qhcjbody{丙} &= \qhcjmerge{丙}{甲} + \qhcjmerge{丙}{乙}\\
  \end{align}
\end{subequations}

\subsection{資料格式說明}
倉頡對於字根的取碼有特殊規定。
若字本身為輔助字根,則不能直接取碼。

%如``氵''為``水''的輔助字根,``工''為``一''的輔助字助。
%在``江''時,其倉頡碼為``水一''
%但若``氵''當獨立字時,則要取碼``卜一''
%若``工''當獨立字時,則要取碼``一中一''

\subsection{注意事項}

\subsection{資料格式說明}
\end{document}
