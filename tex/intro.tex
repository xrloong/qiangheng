
漢字系統,是一個特別的書寫系統。%
它利用數千以至上萬種不同的字形來表達各種意義。%
然而,這些字形卻可以由數百至一千多種字形所構成。
具體的構造方法,則闡述於漢字構造的「六書」中的「會意」及「形聲」。

因為可以藉由組合的方式產生新字,%
於是就產生了兩個特性:「開放」、「數量龐大」。%
「開放」意味著可以依需求造字,%
千餘年來不斷有新的字被造出來,%
這又導致了「數量龐大」的這個特性。%
目前累積已有約十萬字(包含日本、韓國及越南所造漢字)。%
而這兩個特性則都對漢字的資訊化產生了了一定程度的影響。

通常一種語言要資訊化,要經過以下的步驟:
\begin{enumerate}
\item[一、]搜集該語言所用到的字形。\\
搜集到的字形所成的集合稱做字符集或字集。%
但因漢字的開放特性,不可能將所有字都搜集。

\item[二、]賦予每個字形一個獨立的編號(專業的術語稱作編碼)。\\
因為電腦並擅長處理的是數值而非二維的圖形,所以需要對每個字形編號。%
於是,原本對人而言的文字的處理,如儲存、搜尋、排序或比對,%
對電腦而言就是儲存、搜尋、排序或比對數值。%
但因為不可能搜集到所有的字,意味著漏掉的字將無法資訊化。

\item[三、]產生字形檔,要為每個字形準備相對應的圖。\\
電腦可以很輕易地處理數值,但人不行。%
因此,當電腦處理完數值後,要將結果呈現給人時,需要轉換成人類所能理解的形式。%
由於「數量龐大」的原因,要為漢字產生一種字型往往需要耗費大量的人力。%
而如果字符集改變,如新造了或搜集到了一個字,則每種字型再造形。%
為了能將大量的工作轉為自動化,就有了「動態組字」的研究出現。%
「動態組字」的目標在於,利用漢字組字的特性,能動態的將字形產生出來。%
即我們只要準備一些字根的字形,而電腦自動幫我們產生所有漢字的字形。

\item[四、]定義輸入法,為每個字形生成輸入碼。\\
大部分語言所定義的字形數都不像漢字系統採取如此龐大的數量,%
因此在輸入時,通常是使用特殊的鍵盤來容納所有字形,且按壓一個鍵將產生一個字形。
然而漢字系統無法採用此種方式,因此許多字形輸入法被開發出來。%
亦即,利用一組「按鍵序列」來產生一個字。%
然而,因為「數量龐大」的特性,要為一個字形產生一組輸入法,往往也要耗費大量的人力。%
又如新造了或搜集到了一個字,則每種字形都要再次為此字形拆碼。%
為了能將大量的工作轉為自動化,就是本計劃的目標,可以視為「動態拆碼」。

\end{enumerate}

而瑲珩的目標則是希望能實現「動態組字」及「動態拆碼」。

